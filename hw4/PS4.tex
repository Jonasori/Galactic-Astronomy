\documentclass[12pt]{article}
\usepackage[margin=1in]{geometry} 
\usepackage{amsmath,amsthm,amssymb,amsfonts}
\usepackage{graphicx}
 
\newcommand{\N}{\mathbb{N}}
\newcommand{\Z}{\mathbb{Z}}
 
\newenvironment{problem}[2][Problem]{\begin{trivlist}
\item[\hskip \labelsep {\bfseries #1}\hskip \labelsep {\bfseries #2.}]}{\end{trivlist}}
%If you want to title your bold things something different just make another thing exactly like this but replace "problem" with the name of the thing you want, like theorem or lemma or whatever

\newenvironment{answer}[2][Answer]{\begin{trivlist}
\item[\hskip \labelsep {\bfseries #1}\hskip \labelsep {\bfseries #2.}]}{\end{trivlist}}

\begin{document}
 
%\renewcommand{\qedsymbol}{\filledbox}
%Good resources for looking up how to do stuff:
%Binary operators: http://www.access2science.com/latex/Binary.html
%General help: http://en.wikibooks.org/wiki/LaTeX/Mathematics
%Or just google stuff
 
\title{AST 221: Problem Set 4}
\author{Your Name Goes Here}
\maketitle

\noindent {\bf Due: Thursday, April 4 by midnight.} Late papers are not accepted after April 4 (midnight). If you cannot complete the assignment by then, hand in what you have completed before the deadline. Consider the deadline to be like the boarding time for an airplane, or the deadline for a grant submission to NASA or NSF. If you miss the deadline, you do not get on the airplane, no matter how good your excuse is. If you miss an NSF or NASA deadline, you do not get the grant, no matter how good your project is. The best advice is ... finish early. You can submit multiple times, right up to the deadline. Whatever your latest submission is, when the deadline occurs, is what will be graded.
 
\begin{problem}{1} Make an HR diagram of an open (galactic) cluster based on the Gaia catalogue. You may choose any cluster EXCEPT the Pleiades (since you already did the Pleiades as an exercise last semester in AST 222). Also, please make sure you have made a UNIQUE choice of galactic cluster -- do not use the same cluster as any other classmate. (I suggest posting a cluster sign-up board in the student office to be sure that no one duplicates another person's cluster). You can adjust several parameters in selecting members, including the exact position of the center of your search area and the size of your search box. You can use parallaxes and/or kinematic information (proper motions and/or space motions) to help eliminate field stars. Your objective is to get as clean an HR diagram of your cluster as possible -- ie. containing enough cluster members to clearly define the cluster HR diagram with as few non-members as possible. Do NOT remove stars arbitrarily from the diagram just because they don't fit your pre-conceived notion of what the HR diagram should look like! (That's a bad practice in science ...). 

\end{problem}

\begin{answer}{1}
Your answer goes here. Show your work. Justify your conclusions.
\end{answer}

\begin{problem}{2} Once you have a nice cluster HR diagram, overlay on it a ZAMS with the metallicity appropriate to the cluster. You can use the Dartmouth Stellar Evolution Data Base (linked on the course Moodle page) to obtain model results in the Gaia magnitude system. For the galactic cluster, you can use a metallicity of 0 -- i.e. solar. If it is a globular cluster, you will need to look up the metallicity of the cluster and choose an appropriate ZAMS. Adjust the horizontal location of the ZAMS to account for the foreground reddening of the cluster (which you will have to look up or figure out yourself). Then adust the vertical location of the ZAMS to match the cluster MS and use the amount of adjustment to calculate the distance to the cluster. 

\end{problem}

\begin{answer}{2}
Your answer goes here. Show your work. Justify your conclusions.
\end{answer}
 
\begin{problem}{3} Repeat the process in Problems 1 and 2 for a globular cluster of your choice. Again, please make sure your choice of globular cluster is unique.

\end{problem}

 \begin{answer}{3} 
Your answer goes here. Show your work. Justify your conclusions.
\end{answer}
 
\begin{problem}{4} Discuss the HR diagrams above and your procedures for obtaining them. What features can you clearly see on each figure. Identify well known features such as the MS turnoff, the RG, HB, AGB features and the presence or absence of WDs. Can you see a binary sequence? Can you see blue stragglers? Can you see supergiants? Comment on any other aspects of the HR diagram or the procedure for selecting members and non-members that you find noteworthy.

\end{problem}

\begin{answer}{4}
Your answer goes here. Show your work. Justify your conclusions.
\end{answer}
 
\end{document}