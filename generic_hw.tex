\documentclass[12pt]{article}
%\usepackage[margin=1in]{geometry}
\usepackage[left=2.5cm, right=2.5cm, top=2cm]{geometry}
\usepackage{amsmath, amsthm, amssymb, amsfonts}
\usepackage{scrextend}
\usepackage{graphicx}
\usepackage{multicol}
\usepackage{hyperref}


% Set up stuff to handle Python code nicely.
\usepackage{listings}
\usepackage{color}

\definecolor{codegreen}{rgb}{0,0.6,0}
\definecolor{codegray}{rgb}{0.5,0.5,0.5}
\definecolor{codepurple}{rgb}{0.58,0,0.82}
\definecolor{backcolour}{rgb}{0.95,0.95,0.92}

\lstdefinestyle{mystyle}{
    backgroundcolor=\color{backcolour},
    commentstyle=\color{codegreen},
    keywordstyle=\color{magenta},
    numberstyle=\tiny\color{codegray},
    stringstyle=\color{codepurple},
    basicstyle=\footnotesize,
    breakatwhitespace=false,
    breaklines=true,
    captionpos=b,
    keepspaces=true,
    numbers=left,
    numbersep=5pt,
    showspaces=false,
    showstringspaces=false,
    showtabs=false,
    tabsize=2
}

\lstset{style=mystyle}

\setlength{\columnsep}{0.3in}

\newcommand{\N}{\mathbb{N}}
\newcommand{\Z}{\mathbb{Z}}

\newenvironment{problem}[2][Problem]{\begin{trivlist}
\item[\hskip \labelsep {\bfseries #1}\hskip \labelsep {\bfseries #2.}]}{\end{trivlist}}
%If you want to title your bold things something different just make another thing exactly like this but replace "problem" with the name of the thing you want, like theorem or lemma or whatever

\newenvironment{answer}[2][Answer]{\begin{trivlist}
\item[\hskip \labelsep {\bfseries #1}\hskip \labelsep {\bfseries #2.}]}{\end{trivlist}}

\newcommand\textlcsc[1]{\textsc{\MakeLowercase{#1}}}

% Enable one-column figures in multicol.
\newenvironment{Figure}
  {\par\medskip\noindent\minipage{\linewidth}}
  {\endminipage\par\medskip}


\begin{document}

%\renewcommand{\qedsymbol}{\filledbox}
%Good resources for looking up how to do stuff:
%Binary operators: http://www.access2science.com/latex/Binary.html
%General help: http://en.wikibooks.org/wiki/LaTeX/Mathematics
%Or just google stuff

% \title{AST 221: Problem Set 1}
% \author{Jonas Powell}
% \maketitle


% make title bold and 14 pt font (Latex default is non-bold, 11 pt)
\title{\Large \textbf{Galactic Astronomy: Problem Set X}}

\author{{\rm Jonas Powell, \textit{Wesleyan University}}}


\maketitle


\begin{addmargin}[4em]{4em}
\noindent {\bf Due: Thursday, Feb. 14 by midnight.} Late papers are not accepted. If you cannot complete the assignment, hand in what you have completed before the deadline. Consider the deadline to be like the boarding time for an airplane, or the deadline for a grant submission to NASA or NSF. If you miss the deadline, you do not get on the airplane, no matter how good your excuse is. If you miss an NSF or NASA deadline, you do not get the grant, no matter how good your project is. The best advice is ... finish early. You can submit multiple times, right up to the deadline. Whatever your latest submission is, when the deadline occurs, is what will be graded.
\bigskip \bigskip
\end{addmargin}


% Begin two-column layout
% if using this, change margins (line 3) to like 1cm-ish
%\begin{multicols*}{2}


\begin{problem}{1}
\end{problem}

\begin{answer}{1}

  % A FIGURE
  \begin{figure}[htp]
    \hspace*{\fill}%
    \subcaptionbox{\label{fig:lab1}}{\includegraphics[width=0.48\linewidth]{fig2.pdf}}\hfill%
    \subcaptionbox{\label{fig:lab2}}{\includegraphics[width=0.48\linewidth]{fig1.pdf}}\hfill%
    \hspace*{\fill}%
    \caption{blah}
  \end{figure}



  % A TABLE
  \centerline{\textbf{Results}}
  \smallskip
  \centerline{
  \begin{tabular} {cc}
  \hline \hline
  l, b          & (324.52$^o$, 53.49$^o$) \\
  M$_V$         & 6.19 \\
  E(B-V)        & 0.75 \\
  A$_V$         & 2.25 \\
  d             & 30.48 pc \\
  v$_{space}$   & 2628 km s$^{-1}$ \\
  M$_{bol}$     & 5.9 \\
  T$_e$         & 5040 \\
  L/L$_\odot$   & 0.27 \\
  M/M$_\odot$   & 0.72 \\
  R/R$_\odot$   & 0.55 \\
  \hline
  \end{tabular}
  }

\end{answer}
\bigskip \bigskip



%\end{multicols*}
\vfill\eject
\clearpage


\lstinputlisting[language=Python]{scratch_hw2.py}



\end{document}
